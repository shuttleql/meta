\documentclass{article}

\usepackage[utf8]{inputenc}
\usepackage{enumitem}
\usepackage{graphicx}
\usepackage{tikz, amsmath, amssymb, gensymb}
\usepackage[margin=1in]{geometry}

\title{Project Proposal}
\author{SE464}
\date{\today}

\begin{document}

\begin{titlepage}
\newcommand{\HRule}{\rule{\linewidth}{0.5mm}}

\center

\textsc{\huge University of Waterloo}\\[3cm]
\textsc{\LARGE SE464}\\[1.5cm]
\textsc{\Large Section 001}\\[1.5cm]

\HRule \\[0.75cm]
{ \Huge \bfseries Project Proposal}\\[0.5cm]
\HRule \\[2cm]

\Large Group 25 \\  [8cm]

{\Large \today}\\

\vfill
\end{titlepage}

\noindent\textbf{Project Title:} ShuttleQL (Shuttle Queueing Logistics) \\
\textbf{Team Members:}
\begin{itemize}
  \item Cheng Dong (c9dong)
  \item Zhaotian Fang (z23fang)
  \item Clement Hoang (c8hoang)
  \item Di Sen Lu (dslu)
\end{itemize}

\section{Introduction}
\subsection{Motivation}
Around the world, there are 100s of recreational badminton clubs which host
regular sessions at set locations for members to drop in and play at.
Traditionally, the operation of a badminton club was done manually through the
combined efforts of the clubs executive staff. There are several burdenous
and repetitive tasks that the staff members need to do on a regular basis in
order to keep the club running smoothly. For example,
\begin{itemize}
  \item Registering new club members
  \item Checking in/out members during a club session
  \item Scheduling members into courts for each rotation
\end{itemize}
These tasks negatively impact the productivity of the staff and are prone
to human error.
In addition, there is no effective means for the club execs to communicate
with the members in real-time.
Finally, there is no easy way to get feedback on player performance unless
they hire a coach or ask for critique from a staff member.

\subsection{Idea}
ShuttleQL is an all-in-one badminton club management platform which not only
automates repetitive administrative tasks but also acts as a communication hub
between the club executives and its members. It also offers insights and
analytics into performance of the players in the club.

The platform will be entirely web-based. It will consist of two components:
a club member mobile web based dashboard and an admin desktop based dashboard.
The platform from the club member's perspective will be a mobile web based app
since they are more likely to have access to a phone than a laptop when they
come to session. The admin dashboard will be desktop based since it requires
heavy user interaction which will be easier on a laptop rather than on a phone.

\subsection{Originality}
Although existing software solutions exist for a similar purpose, they are often
incomplete relative to ShuttleQL and don't leverage many of todays technologies.
ShuttleQL aims to satisfy most if not all of the pain points associated with
managing a badminton club. It also applies new concepts such as data analytics and
machine learning into processes such as matchmaking and coaching, which has not
yet been done in existing solutions.

\newpage

\section{Project Properties}
\subsection{Functional}
\begin{enumerate}
  \item Match making
  \item Send notifications to users once they are able to play
  \item Allow executives to choose which users to appear on the courts
  \item Display current players on courts
  \item Show user match history
  \item Track user match-making-rating in ladder matches
\end{enumerate}

\subsection{User Scenarios}
There are two different types of users in our system. The first type of user is the badminton club's executives. They would interact with our system by manually selecting which players to appear on the courts. They would also have access to change any user's information such as name, level, category preference. Although there currently is a way for the executives to manually select players to appear on the courts, our system would benefit the executives by allowing them to change user information faster.

The second type of user is the normal badminton club players. They would interact with our system by viewing on their phones the current players playing on every court. They would also get notifications when it's their turn to play. Furthermore, they are allowed to view their match history and modify their account information such as name, category preference and contact information. The normal badminton club players would benefit from this by viewing their time to play on their phone rather than crowding around a bulletin board to view this information. In addition, they will be notified when it's their time to play rather than constantly walking up to the bulletin board and checking.

\subsection{Non-functional}
The two non-functional properties that our system needs to support are dependability and adaptability. Dependability describes the reliability of the system. There should be no crashes or unexpected software failures. Dependability also describes how accurate the software features meets the specifications. The system should behave exactly as the specifications describe without encountering any software bugs.

Adaptability describes how well the system adapts to new requirements to the software. Every time that a new requirement needs to added to the system, the system must be able to function as expected. For example, the match making system should function as expected even if the maximum number of players on a court change.

\section{Mockups}

\end{document}
